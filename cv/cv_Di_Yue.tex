%%%%%%%%%%%%%%%%%%%%%%%%%%%%%%%%%%%%%%%%%
% Medium Length Professional CV
% LaTeX Template
% Version 2.0 (8/5/13)
%
% This template has been downloaded from:
% http://www.LaTeXTemplates.com
%
% Original author:
% Trey Hunner (http://www.treyhunner.com/)
%
% Important note:
% This template requires the resume.cls file to be in the same directory as the
% .tex file. The resume.cls file provides the resume style used for structuring the
% document.
%
%%%%%%%%%%%%%%%%%%%%%%%%%%%%%%%%%%%%%%%%%

%----------------------------------------------------------------------------------------
%	PACKAGES AND OTHER DOCUMENT CONFIGURATIONS
%----------------------------------------------------------------------------------------

\documentclass{resume} % Use the custom resume.cls style

\usepackage[left=0.75in,top=0.6in,right=0.75in,bottom=0.6in]{geometry} % Document margins
\usepackage{amsmath,amsthm,amsfonts,amssymb}
\newcommand{\tab}[1]{\hspace{.2667\textwidth}\rlap{#1}}
\newcommand{\itab}[1]{\hspace{0em}\rlap{#1}}
\name{Di Yue} % Your name
% \address{xxx} % Your address
\address{Homepage: \href{https://tnediserp.github.io}{https://tnediserp.github.io}} % Your secondary addess (optional)
% \address{Tel: (+86)18380578521} % Tel
\address{Email: diyue2003@gmail.com \\ Tel: (+86)18380578521} % Email

\begin{document}

%----------------------------------------------------------------------------------------
%	EDUCATION SECTION
%----------------------------------------------------------------------------------------

\begin{rSection}{Education}

{\bf Peking University} \hfill {Beijing, China} 
\\ Bachelor of Science (Summa Cum Laude) \hfill {\em September 2021 - July 2025}
\\ School of Electronics Engineering And Computer Science
\\ GPA: 3.819/4.0

%Minor in Linguistics \smallskip \\
%Member of Eta Kappa Nu \\
%Member of Upsilon Pi Epsilon \\


\end{rSection}
%----------------------------------------------------------------------------------------
%	TECHNICAL STRENGTHS SECTION
%----------------------------------------------------------------------------------------

\begin{rSection}{Research Interests}{}
Theoretical Computer Science, Approximation Algorithm, High-dimensional Computational Geometry, Metric Embeddings.
\end{rSection}

\begin{rSection}{Publications}{(In theoretical computer science, authors are listed in alphabetical order.)}
    % \begin{center}
    %     {\bf Submitted}
    % \end{center}
    \begin{pubSubsection}{Dimension Reduction for Clustering: The Curious Case of Discrete Centers}
        \item Shaofeng H.-C. Jiang, Robert Krauthgamer, Shay Sapir, Sandeep Silwal, \textbf{Di Yue}.
        \item Manuscript.
    \end{pubSubsection}

    \begin{pubSubsection}{Near-Optimal Dimension Reduction for Facility Location}
        \item Lingxiao Huang, Shaofeng H.-C. Jiang, Robert Krauthgamer, \textbf{Di Yue}.
        \item In Proceedings of \emph{The 57th ACM Symposium on Theory of Computing (STOC 2025)}.
        % \href{https://arxiv.org/abs/2411.05432}{\texttt{arXiv:2411.05432}}.
    \end{pubSubsection}
\end{rSection}

\begin{rSection}{Research Experience}{}
    \begin{rSubsection}{Research Assistant in Shaofeng Jiang's Lab}{April 2025 - Present}{Advisor: Shaofeng Jiang}{Peking University, China}
        \item Studying dimension reduction for geometric problems.
    \end{rSubsection}

    \begin{rSubsection}{Visiting Student at Weizmann Institute of Science.}{August 2024 - September 2024}{Advisor: Robert Krauthgamer}{Weizmann Institute of Science, Israel}
        \item Studied dimension reduction for MST and Steiner tree problems.
        \item Gave a new proof of dimension reduction for MST, using target dimension $m = O(\varepsilon^{-2} \mathrm{ddim} \cdot \log\log \Delta)$.
        \item Gave a talk on our recent \emph{uniform facility location (UFL)} work in the algorithm seminar.
    \end{rSubsection}

    \begin{rSubsection}{Research Assistant in Shaofeng Jiang's Lab}{July 2023 - March 2025}{Advisor: Shaofeng Jiang}{Peking University, China}
       \item Studied dimension reduction for \emph{uniform facility location (UFL)}. 
       Proved that target dimension $m = \tilde{O}(\varepsilon^{-2} \mathrm{ddim})$ suffices to $(1+\varepsilon)$-approximate the optimal value of UFL on high-dimensional inputs whose \emph{doubling dimension} is bounded by $\mathrm{ddim}$.
       \item Proposed the first PTAS for Euclidean UFL on doubling subsets, where the facilities are allowed to lie in the (high-dimensional) ambient space $\mathbb{R}^d$. 
       Generalized our PTAS to doubling metrics without vector representations, improving previous results in [Cohen-Addad et al., JACM 2021].
       \item Wrote the technical sections of the paper, and helped with some parts in the introduction.
    \end{rSubsection}

    \begin{rSubsection}{Research Rotation in Computer Science Department}{January 2023 - June 2023}{Advisor: Shaofeng Jiang, Tianren Liu}{Peking University, China}
        \item Algorithm: Studied dimension reduction for Euclidean diameter. 
        Proved that target dimension $m = O(\varepsilon^{-2} \mathrm{ddim})$ suffices to $(1+\varepsilon)$-approximate the diameter of a high-dimensional doubling subset whose \emph{doubling dimension} is bounded by $\mathrm{ddim}$.
        % \item This result immediately implies a streaming algorithm that approximates diameter.
        \item Cryptography: Studied \emph{Private Information Retrieval (PIR)}.
        Wrote a survey on the upper and lower bounds for PIR.
    \end{rSubsection}
\end{rSection}

\begin{rSection}{Teaching Experience}{}
    \begin{teachSubsection}{Algorithm Design and Analysis, Teaching Assistant}{Spring 2025}{Lecturer: Xiaolin Wang, Tingting Jiang}{Peking University}
        %
    \end{teachSubsection}
\end{rSection}

\begin{rSection}{Honours and Awards}{}
    Top 10 Undergraduate Thesis, EECS, Peking University \dotfill{2025}
    \\ Academic Excellence Award \dotfill{2023}
    \\ Second Class Scholarship of Peking University \dotfill{2022}
    \\ Merit Student \dotfill{2022}
\end{rSection}

\end{document}